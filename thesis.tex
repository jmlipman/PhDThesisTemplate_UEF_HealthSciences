%% Example main file for UEF thesis
%
\documentclass[b5paper,12pt]{memoir}  % memoir class documentation: https://www.ctan.org/pkg/memoir
%
%%%%%%%%%%%%%%%%%%%%% Packages %%%%%%%%%%%%%%%%%%%%%%
%
% language and font encoding
%
\usepackage[english]{babel} % https://www.ctan.org/pkg/babel
\usepackage[utf8]{inputenc} % https://www.ctan.org/pkg/inputenc
\usepackage{ae}             % https://www.ctan.org/tex-archive/fonts/ae/
%
% Specifying fonts
%
\usepackage{tgpagella}    % Sets TeX Gyre Pagella as the default (serif) font for the document.
\usepackage[sc]{mathpazo} % Used for mathematical typesetting with the Palatino fonts.
%
\usepackage{blindtext} % This is used in the example-file just to generate some text.
                       % You can remove this in the real case.
%
% Specifying citing matters
%
%\usepackage{cite}
\usepackage{natbib}
%
% For figures
%
\usepackage[dvips]{graphicx} % https://www.ctan.org/pkg/graphicx
\usepackage{epsfig}          % https://www.ctan.org/pkg/epsfig
\usepackage{xcolor}          % https://www.ctan.org/pkg/xcolor
%
% For math
%
\usepackage{amsfonts} % https://www.ctan.org/pkg/amsfonts
\usepackage{amsmath}  % https://www.ctan.org/pkg/amsmath
\usepackage{amsthm}   % https://www.ctan.org/pkg/amsthm
\usepackage{amssymb}  % is a \LaTeX{} option that defines symbol names for all the math symbols in amsfonts
\usepackage{bbm}
\usepackage{svg}
%
% Miscellaneous
%
\usepackage{url}      % https://www.ctan.org/tex-archive/macros/latex/contrib/url/



% MIGUEL'S MODIFICATIONS
\DisemulatePackage{setspace}
\usepackage{setspace}
%\onehalfspacing
\singlespacing


%\footskip .25in



\usepackage{siunitx} % For writing "micro" with Greek letters.
\usepackage{multirow}
\definecolor{red}{rgb}{1, 0, 0} % Miguel: necessary for the line below
\newcommand{\revise}[1]{{\color{red}{\textbf{#1}}}} % Miguel: Custom command to highlight things
\renewcommand\familydefault{\sfdefault} % Miguel: Change the font of the entire document to SF
%\setlength\bibhang{0in} % Miguel: Removes the indentation (all lines except the first one of each entry) of the bibliography.
\usepackage[hang,flushmargin]{footmisc} % Miguel: Removes indentation in the footnote
\usepackage{lipsum}  % Miguel: lovely lorem ipsum package
\usepackage{chngcntr} % Miguel: For later using the command "counterwithout" to change fig/tab numbering
\hyphenpenalty=10000 % Miguel: avoid breaking words

\usepackage{tocloft} % Miguel: For the line below
\renewcommand{\cftchapterleader}{\textbf{\cftdotfill{0}}} % Miguel: add dots in chapters (ToC)


\DeclareMathOperator*{\argmax}{arg\,max}
\DeclareMathOperator*{\argmin}{arg\,min}

%
  %  The file uefthesis.sty contains relevant definitions for the thesis.
  %  This file should not be modified!!
  %  If you want to use your own definitions you should prepare your own mystyle.sty file
%
\usepackage{UEF_thesis}

% Miguel: This is for removing text justification. It seems that it doesn't need to be mandatory.
\usepackage{ragged2e}
\setlength{\RaggedRightParfillskip}{.25\textwidth plus 1fil}
\setlength{\RaggedRightRightskip}{0pt plus .1\textwidth}
\setlength{\RaggedRightParindent}{2em}
\RaggedRight

%
%%%%%%%%%%%%%%%%%%%%% Document %%%%%%%%%%%%%%%%%%%%%
%
\usepackage{pdfpages}

\usepackage{enumitem}
\let\svitem\item%
\def\mybox#1{\makebox[2cm][l]{#1}}
\newenvironment{leftitemize}
{\renewcommand\item[1][$\bullet$]{\svitem[\mybox{##1}]}%
  \begin{itemize}[leftmargin=\dimexpr2cm+\labelsep]}{\end{itemize}}

\begin{document}
%
%\frontmatter
%
%  File opening.tex contains the layout for title page, abstract etc.
%  You should modify it suitably.
%
%\iffalse
%% Example file for the frontmatter
%
% This file contains layout setting examples for front page, abstract, etc.
%
% Note: for the university logo the file UEF_logo.eps is needed.
%
\title{{Title}}
\subtitle{Subtitle} % Commend this line if you don't have a subtitle
\author{Your Name}
%
% Macro for frontpage: volume, place, year, department, and place and time of the examination
%

\makefrontpage{\revise{77}}{}{}
{A.I. Virtanen Institute for Molecular Sciences}
{\revise{SN201 Auditorium, Kuopio \\ on September 6\textsuperscript{th}, 2022, at 12:00}}
 {}{} % if like to add an figure here replace empty brackets with 
      % {your_file_name.eps}{figure_height}, where figure_height is in mm, e.g. 50mm.   
%
\clearpage

%
% Page for printing place, editors, and identification numbers,
% edit the lines below as needed
%
\begin{center}
\vspace*{\fill}
\renewcommand{\baselinestretch}{1.4}
Series Editors \\
Professor Tomi Laitinen, M.D., Ph.D. \\
Institute of Clinical Medicine, Clinical Physiology and Nuclear Medicine \\
Faculty of Health Sciences\\[0.7cm]

Professor Tarja Kvist, Ph.D. \\
Department of Nursing Science \\
Faculty of Health Sciences \\[0.7cm]

Professor Ville Leinonen, M.D., Ph.D. \\
Institute of Clinical Medicine, Neurosurgery \\
Faculty of Health Sciences \\[0.7cm]

Professor Tarja Malm, Ph.D. \\
A.I. Virtanen Institute for Molecular Sciences \\
Faculty of Health Sciences \\[0.7cm]

Lecturer Veli-Pekka Ranta, Ph.D. \\
School of Pharmacy \\
Faculty of Health Sciences \\[0.7cm]
  %Editors:  Pertti Pasanen, Pekka Toivanen,\\
  %Jukka Tuomela, and Matti Vornanen\\[1cm]
  
  Distributor:\\
  University of Eastern Finland\\
  Kuopio Campus Library \\
  P.O.Box 1627 \\
  FI-70211 Kuopio, Finland \\
  http://www.uef.fi/kirjasto\\[0.7cm]
  
  PunaMusta Oy \\
  Joensuu, 2022\\[1cm]
 
  ISBN: \revise{978-952-61-4622-5} (print/nid.)\\
  ISBN: \revise{978-952-61-4623-2} (PDF)\\
  ISSNL: \revise{1798-5706} \\
  ISSN:  \revise{1798-5706} \\
  ISSN:  \revise{1798-5714} (PDF)
\end{center}
%\clearpage
%
% Page for supervisors', reviewers' and opponent's information,
% edit the lines below as needed
%
\begin{tabular}{p{45mm}l}
  Author's address: & A.I. Virtanen Institute for Molecular Sciences \\
              & University of Eastern Finland\\ % or Institute of Photonics
              & KUOPIO \\
              & FINLAND \\ [5mm]
  Doctoral programme: & Doctoral Programme in Molecular Medicine \\ [5mm]
  Supervisors:& Professor \revise{Name Surname}, Ph.D\\
              & A.I. Virtanen Institute for Molecular Sciences \\
              & University of Eastern Finland \\
              & KUOPIO \\
              & FINLAND \\[5mm]
              
              & Professor \revise{Name Surname}, Ph.D. \\
              & A.I. Virtanen Institute for Molecular Sciences \\
              & University of Eastern Finland \\
              & KUOPIO \\
              & FINLAND \\[5mm]
              
  Reviewers:  & Assistant Professor \revise{Name Surname}, Ph.D. \\
              & Institute of  \\
              & University of  \\
              & City \\
              & Country \\[5mm]
              & Senior Lecturer \revise{Name Surname}, Ph.D. \\
              & Center for  \\
              & University of  \\
              & Department of (affiliation 2) \\
              & University Hospital \\
              & City \\
              & Country \\[5mm]
  Opponent:   & Professor \revise{Name Surname}, Ph.D.\\
              & Department of  \\
              & University of \\
              & City \\
              & Country
\end{tabular}
%
\cleardoublepage
%\clearpage
%
% Abstract page with heading
%
\pagenumbering{arabic}
\setcounter{page}{7}
\makeabstracthead{Kuopio}{2022} % place and year
%
\addcontentsline{toc}{chapter}{ABSTRACT}
\section*{ABSTRACT}
%

Abstract goes here.


%
\par
\vspace{5mm}
\noindent
%\textit{\textbf{Universal Decimal Classification:} \revise{53.084.85, 535.3, 535.4, 681.7.02} \\
%
% select (or add) a suitable code
%
%\textbf{MSC 2010:} 35J20,  47G10, 30F15 \\   % see http://www.ams.org/msc/msc2010.html
%\textbf{PACS 2010:} 42.40.Eq, 42.40.Jv  \\   % see https://publishing.aip.org/publishing/pacs/pacs-2010-regular-edition
%\textbf{OCIS codes:} \revise{050.1960, 130.4815}  \\   % see https://www.osapublishing.org/submit/ocis/
%
%\textbf{Keywords:} \revise{Diffraction gratings, optical switching devices, partial differential equations, complex analysis}}
Medical Subject Headings: \revise{Keywords}.

\vspace{5mm}
\noindent
Yleinen suomalainen ontologia: \revise{Keywords in Finnish}
%
\clearpage
%
\addcontentsline{toc}{chapter}{ACKNOWLEDGEMENTS}
\section*{{\fontsize{16pt}{0}\selectfont ACKNOWLEDGEMENTS}}



\vspace{0.5cm}
%

Acknowledgements go here.


%
\par
\vspace{9mm}
\noindent
Kuopio, September 2022    \\[3mm]
\begin{figure}[h]
    \includegraphics[width=0.4\linewidth]{figures/signature.png}
\end{figure}
      
\indent\hspace{6mm}\textit{\theauthor}
\vfill

\newpage
\par

\phantom{.}

\vspace{5cm}
\noindent
\lq All we have to decide is what to do with the time that is given us\rq


\vspace{1cm} \noindent 
\textit{Gandalf, The Lord of the Rings: The Fellowship of the Ring}
\vfill

%
\cleardoublepage
%
% Page for publications - in the case of monograph possible publications
% can be listed at the end of introduction.
%
\section*{{\fontsize{16pt}{0}\selectfont \normalfont LIST OF ORIGINAL PUBLICATIONS}}

%
\vspace{0.1in}
This dissertation is based on the following original publications:
%
\begin{enumerate}
%
\item[\textbf{\myp{1}}]
Valverde, J.M., Shatillo, A., De Feo, R., Gröhn, O., Sierra, A. and Tohka, J. Automatic rodent brain mri lesion segmentation with fully convolutional networks. In \textit{International Workshop on Machine Learning in Medical Imaging} (pp. 195-202). \textit{Springer, Cham} (2019).

\item[\textbf{\myp{2}}]
Valverde, J.M., Shatillo, A., De Feo, R., Gröhn, O., Sierra, A. and Tohka, J. Ratlesnetv2: a fully convolutional network for rodent brain lesion segmentation. \textit{Frontiers in neuroscience}, (14) 610239 (2020).

\item[\textbf{\myp{3}}]
Valverde, J.M., Shatillo, A., De Feo, R. and Tohka, J. Automatic cerebral hemisphere segmentation in rat MRI with lesions via attention-based convolutional neural networks. arXiv:2108.01941. Submitted to \textit{NeuroInformatics} (in revision).

\item[\textbf{\myp{4}}]
Valverde, J.M., Imani, V., Abdollahzadeh, A., De Feo, R., Prakash, M., Ciszek, R. and Tohka, J. Transfer learning in magnetic resonance brain imaging: A systematic review. \textit{Journal of imaging}, 7(4), p.66 (2021).

\item[\textbf{\myp{5}}]
Valverde, J.M. and Tohka, J. Region-wise Loss for Biomedical Image Segmentation. arXiv:2108.01405. Submitted to \textit{Pattern Recognition} (in revision).
%
\end{enumerate}
%
\vspace{2mm}
\noindent
The publications were adapted with the permission of the copyright owners.
The software we developed in our works is open source and publicly available:

\begin{itemize}
\item RatLesNetv2 for lesion segmentation in rodent MR images: \\ https://github.com/jmlipman/RatLesNetv2
\item MedicDeepLabv3+ for cerebral hemisphere segmentation in rodent MR images: https://github.com/jmlipman/MedicDeepLabv3plus
\item Region-wise loss function: https://github.com/jmlipman/RegionWiseLoss
\end{itemize}
\vspace{20mm}
%
\section*{AUTHOR'S CONTRIBUTION}
%
% Only for thesis consisting on articles.
%
Your contributions

%
\vfill










%
%\hfill % Miguel: so that the empty page includes a number
\cleardoublepage

%
% Create the table of contents
\renewcommand{\contentsname}{CONTENTS} % Change title of ToC
\tableofcontents*

%
  %% If you like to have a list of figures, list of tables, or list of symbols:
  %
  % \clearpage
  % \listoffigures
  % \clearpage
  % \listoftables
  % \clearpage
\noindent
{\Large\sffamily ABBREVIATIONS}
    
\vspace{0.1in}

\begin{leftitemize}
\setlength\itemsep{5mm}
\item[2D] Two Dimensional
\item[3D] Three Dimensional
\end{leftitemize}
%\fi
%
%\mainmatter
%
%% Here is the main text in the file examples.tex.
%
% WORD TEMPLATE: https://kamu.uef.fi/en/tietopankki/guides-and-instructions/theses/

\counterwithout{figure}{chapter} % Miguel: Changes the number of figures and tables, from
\counterwithout{table}{chapter} % 1.1, 1.2... to 1, 2, etc.
\counterwithout{equation}{chapter}
\chapter{Chapter}

\section{Section}

\subsection{Subsection}

\subsubsection{Subsubsection}

Lorem ipsum.
\include{2_relatedwork}
\include{3_papers}
\include{4_discussion}

\renewcommand{\bibsection}{\section*{\Large References}} % Miguel: Change "bibliography" to "references"
\label{bibbib}
\renewcommand\bibpreamble{\vspace{1\baselineskip}} % choose a suitable vert. skip
\setlength{\bibsep}{0pt} % no extra vert. space between bib items


\bibliographystyle{agsm}
{\fontsize{11pt}{0}
\bibliography{biblio}}

%
%  If you have appendices then they can be included here
%
%%% Exapmle file for appendices
%
% Appendices can be formulated quite freely.
% Each appendix is like a chapter. 
% These chapters will be 'numbered' with alphabets, A, B, C, etc.
%
\appendix % defines the beginning of appendices
% 
\chapter{EQUATIONS}
%
\section{First equation}
%
\begin{equation}
\eta=\int_{-\infty}^\infty f(x)\exp\left(\textrm{i} 2\pi mx/d\right)d x
\end{equation}
%
\blindmathpaper
%
\section{Second equation}
%
\begin{equation}
n_3\sin\theta_m=n_1\sin\theta_\mathrm{in}
\end{equation}
%
\blindmathpaper








%
% If you are including articles in your thesis, the cover pages are needed
% ADD THIS LATER...
%\include{coverpages}
%

% ARTICLES INCLUDED IN THE THESIS

\cleardoublepage

\pagenumbering{gobble}
\begin{vplace}[0.7]
\centering
{\fontsize{16pt}{0}\selectfont ORIGINAL PUBLICATIONS (I-V)}
\end{vplace}

\cleardoublepage
\begin{vplace}[0.7]
\centering
\textbf{I} \\[1cm]

\textbf{Automatic Rodent Brain MRI Lesion Segmentation with \\ Fully Convolutional Networks}\\[1cm]

Juan Miguel Valverde, Artem Shatillo, Riccardo De Feo, Olli Gröhn, \\ Alejandra Sierra, Jussi Tohka \\[1cm]

International Workshop on Machine Learning in Medical Imaging, 195-202, 2019 \\[1cm]

Printed with the kind permission of Springer Nature
\end{vplace}

\cleardoublepage

\includepdf[pages=-]{articles/article.pdf}


\cleardoublepage
\begin{vplace}[0.7]
\centering
\textbf{II} \\[1cm]

\textbf{RatLesNetv2: A Fully Convolutional Network for Rodent Brain Lesion Segmentation}\\[1cm]

Juan Miguel Valverde, Artem Shatillo, Riccardo De Feo, Olli Gröhn, \\ Alejandra Sierra, Jussi Tohka \\[1cm]

Frontiers in neuroscience 14: 610239, 2020
\end{vplace}


\cleardoublepage


\includepdf[pages=-]{articles/article.pdf}

\cleardoublepage
\begin{vplace}[0.7]
\centering
\textbf{III} \\[1cm]

\textbf{Automatic cerebral hemisphere segmentation in rat MRI with lesions via attention-based convolutional neural networks}\\[1cm]

Juan Miguel Valverde, Artem Shatillo, Riccardo De Feo, Jussi Tohka \\[1cm]

arXiv:2108.01941, 2022
\end{vplace}

\cleardoublepage


\includepdf[pages=-]{articles/article.pdf}


\cleardoublepage
\begin{vplace}[0.7]
\centering
\textbf{IV} \\[1cm]

\textbf{Transfer Learning in Magnetic Resonance Brain Imaging: a Systematic Review}\\[1cm]

Juan Miguel Valverde, Vandad Imani, Ali Abdollahzadeh, Riccardo De Feo, Mithilesh Prakash, Robert Ciszek, Jussi Tohka \\[1cm]

Journal of imaging 7 (4), 66, 2021
\end{vplace}

\cleardoublepage


\includepdf[pages=-]{articles/article.pdf}


\cleardoublepage
\begin{vplace}[0.7]
\centering
\textbf{V} \\[1cm]

\textbf{Region-wise Loss for Biomedical Image Segmentation}\\[1cm]

Juan Miguel Valverde, Jussi Tohka \\[1cm]

arXiv:2108.01405, 2022
\end{vplace}

\cleardoublepage


\includepdf[pages=-]{articles/article.pdf}

\end{document}

